\section{15. tétel}

\begin{definicio}{EULER-FÉLE $\varphi$-FÜGGVÉNY}
Ha $n \geq 2$ egész, akkor az $1,\ldots,n-1$ számok között n-hez relatív prímek számát $\varphi(n)$-el jelöljük. Az $n\mapsto\varphi(n)$ függvényt Euler-féle $\varphi$ függvénynek nevezzük.
\end{definicio}

\begin{tetel}{EULER-FÉLE $\varphi$-FÜGGVÉNYRE KÉPLET}
Legyen az $n > 1$ egész kanonikus alakja $n = p_1^{\alpha_1} \cdot \ldots \cdot p_k^{\alpha_k}$ Ekkor $$\varphi(n) = \left(p_1^{\alpha_1} - p_1^{\alpha_1-1}\right) \cdot \left(p_2^{\alpha_2} - p_2^{\alpha_2-1}\right) \cdot \ldots \cdot
\left(p_k^{\alpha_k} - p_k^{\alpha_k-1}\right)$$
\end{tetel}

\begin{bizonyitas}{}
130-131. oldal Szeszlér-jegyzet.
\end{bizonyitas}

\begin{definicio}{REDUKÁLT MARADÉKRENDSZER}
Az $R = \{c_1, \ldots, c_k\}$ számhalmaz redukált maradékrendszer modulo m, ha a következő feltételeknek eleget tesz:
\begin{itemize}
\item ($c_i, m$) = 1 minden i = 1, $\ldots$, k esetén;
\item $c_i \not \equiv c_j$ (mod m) bármely $i \neq j, 1 \leq i, j \leq k$ esetén;
\item $k = \varphi(m).$
\end{itemize}
\end{definicio}

\begin{tetel}{REDUKÁLT MARADÉKRENDSZER}
Legyen $R = \{c_1, \ldots, c_k\}$ redukált maradékrendszer modulo m és legyen tetszőleges egész, melyre (a,m) = 1. Ekkor az $R' = \{a\cdot c_1, \ldots, a \cdot c_k\}$ halmaz szintén redukált maradékrendszer modulo m.
\end{tetel}

\begin{bizonyitas}{}
132. oldal Szeszlér-jegyzet.
\end{bizonyitas}

\begin{tetel}{EULER-FERMAT}
Ha az a és m $\geq$ 2 egészekre (a,m) = 1, akkor $a^{\varphi(m)} \equiv 1$ (mod m).
\end{tetel}

\begin{bizonyitas}{}
132-133. oldal Szeszlér-jegyzet.
\end{bizonyitas}

\begin{tetel}{"KIS" FERMAT-Tétel}
Ha p prím és a tetszőleges egész, akkor $a^p \equiv a$ (mod p).
\end{tetel}

\begin{bizonyitas}{}
133. oldal Szeszlér-jegyzet.
\end{bizonyitas}

A tételhez hozzátartozik diofantikus illetve két kongruenciából álló kongruenciarendszerek megoldása is, konkrét példán.
