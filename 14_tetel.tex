\documentclass[]{article}
\usepackage{lmodern}
\usepackage{amssymb}
\usepackage{amsmath}
\usepackage{polyglossia}
\usepackage{listings}
\usepackage{tcolorbox}
\usepackage{etoolbox}
\usepackage{setspace}
\usepackage{framed}
\usepackage[a4paper,margin=2cm,footskip=.5cm]{geometry}
\newcommand{\R}{\mathbb{R}}
\newcommand{\Rn}[1]{$\mathbb{R}^{#1}$}
\newcommand{\Und}[1]{\underline{#1}}
\definecolor{shadecolor}{gray}{0.9}
%opening 
\title{Bevezetés a Számításelméletbe 1.\\{\large 14. tétel}}
\author{Hegyi Zsolt}
\begin{document}
\maketitle{}
\begin{framed}
LINEÁRIS KONGRUENCIÁK MEGOLDHATÓSÁGA Tétel: Az $a \cdot x \equiv b$ (mod m) lineáris kongruencia akkor és csak akkor megoldható, ha $(a,m)|b$. Ha pedig ez a feltétel teljesül, akkor  $a \cdot x \equiv b$ (mod m) megoldásainak a száma modulo m (a,m)-val egyenlő.
\end{framed}
\begin{leftbar}
Biz: 124. oldal Szeszlér-jegyzet.
\end{leftbar}
\begin{framed}
EUKLIDESZI ALGORITMUS: Bemenet: a és m (0 < a < m) | Kimenet: (a,m).
\begin{description}
\item[1. lépés:]m-et maradékosan osztjuk a-val, megkapva a maradékot, felírjuk őket a következő módon:
$$a = b\cdot q_1 + r_1$$
\item[2. lépés:]az a-t eloszjuk a kapott maradékkal:
$$b = r_1\cdot q_2 + r_2$$
\item \vdots
\item[i. lépés:] az (i-2). lépésben kapott maradékot elosztjuk az (i-1).-ben kapottal.
$$r_{i-2} = r{i-1} + r_i$$
\item[Utolsó lépés] Akkor érünk el ide, ha $r_i = 0$, ekkor $r_{i-1}$ lesz a legnagyobb közös osztó.
\end{description}
\end{framed}
\begin{framed}
EUKLIDESZI ALGORITMUS Tétel: Az Euklideszi algoritmus végrehajtása után $r_k = (a,m)$.
\end{framed}
\begin{leftbar}
Biz: 142. oldal Szeszlér-jegyzet.
\end{leftbar}
\begin{framed}
EUKLIDESZI ALGORITMUS LÉPÉSSZÁMA Tétel: Az Euklideszi algoritmus polinomiális időben lefut és legfeljebb $2 \cdot [\log_2a]$ maradékos osztás után megáll.
\end{framed}
\begin{leftbar}
Biz: 142. oldal Szeszlér-jegyzet.
\end{leftbar}
A tételhez hozzá tartozik az Euklidesz algoritmus alkalmazása lineáris kongruenciák megoldásához, konkrét példán.
\end{document}